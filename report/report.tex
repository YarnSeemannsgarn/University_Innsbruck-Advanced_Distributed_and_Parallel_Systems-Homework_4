%%%%%%%%%%%%%%%%%%%%%%%%%%%%%%%%%%%%%%%%%
% University/School Laboratory Report
% LaTeX Template
% Version 3.1 (25/3/14)
%
% This template has been downloaded from:
% http://www.LaTeXTemplates.com
%
% Original author:
% Linux and Unix Users Group at Virginia Tech Wiki 
% (https://vtluug.org/wiki/Example_LaTeX_chem_lab_report)
%
% License:
% CC BY-NC-SA 3.0 (http://creativecommons.org/licenses/by-nc-sa/3.0/)
%
%%%%%%%%%%%%%%%%%%%%%%%%%%%%%%%%%%%%%%%%%

%----------------------------------------------------------------------------------------
%	PACKAGES AND DOCUMENT CONFIGURATIONS
%----------------------------------------------------------------------------------------

\documentclass{article}

%\usepackage[version=3]{mhchem} % Package for chemical equation typesetting
%\usepackage{siunitx} % Provides the \SI{}{} and \si{} command for typesetting SI units
\usepackage{graphicx} % Required for the inclusion of images
%\usepackage{natbib} % Required to change bibliography style to APA
%\usepackage{amsmath} % Required for some math elements 
\usepackage{listings}
\lstset{
  breaklines=true,
  basicstyle=\scriptsize,
  columns=fullflexible
}

\usepackage{tikz}

\setlength\parindent{0pt} % Removes all indentation from paragraphs

\renewcommand{\labelenumi}{\alph{enumi}.} % Make numbering in the enumerate environment by letter rather than number (e.g. section 6)

%\usepackage{times} % Uncomment to use the Times New Roman font

%----------------------------------------------------------------------------------------
%	DOCUMENT INFORMATION
%----------------------------------------------------------------------------------------

\title{Report: Homework 3 - Grid Computing}% Title

\author{Jan \textsc{Schlenker}} % Author name

\date{\today} % Date for the report

\begin{document}

\maketitle % Insert the title, author and date

\begin{center}
\begin{tabular}{l l}
Instructor: & Dipl.-Ing. Dr. Simon Ostermann \\
Parts solved of the sheet: & Tasks 1-3 \\
Total points: & 10
\end{tabular}
\end{center}

% If you wish to include an abstract, uncomment the lines below
% \begin{abstract}
% Abstract text
% \end{abstract}

%----------------------------------------------------------------------------------------
%	SECTION 1
%----------------------------------------------------------------------------------------

\section{How to run the programme}

First of all extract the archive file \texttt{homework\_4.tar.gz}:

\begin{lstlisting}[language=bash, deletekeywords={cd}]
  $ tar -xzf homework_4.tar.gz
  $ cd homework_4
\end{lstlisting}

Afterwards move/copy the binary files \texttt{gm} and \texttt{povray} to the \texttt{bin/} directory and the files \texttt{scherk.args}, \texttt{scherk.ini} and \texttt{scherk.pov} to the \texttt{inputdata/} directory:

\begin{lstlisting}[language=bash]
  $ cp <gm-file-path> <povray-file-path> bin/
  $ cp <scherk-files-dir>/scherk* inputdata/
\end{lstlisting}

At last run the gridRender.sh script:

\begin{lstlisting}[language=bash]
  $ ./gridRender.sh
\end{lstlisting}

%\begin{center}\ce{2 Mg + O2 -> 2 MgO}\end{center}

% If you have more than one objective, uncomment the below:
%\begin{description}
%\item[First Objective] \hfill \\
%Objective 1 text
%\item[Second Objective] \hfill \\
%Objective 2 text
%\end{description}

%----------------------------------------------------------------------------------------
%	SECTION 2
%----------------------------------------------------------------------------------------

\section{Programme explanation}
The files of the the programme are structured as follows:

\begin{itemize}
\item The \texttt{\textbf{gridRender.sh}} script contains the main programme
\item The \texttt{\textbf{bin}} directory contains the binaries \texttt{povray} and \texttt{gm} which will be copied to the grid instances
\item The \texttt{\textbf{inputdata}} directory contains the necessary files for the \texttt{povray} binary which will be copied to the grid instances
\end{itemize}

Below is the programme explanation task by task:

\begin{itemize}

\item \textbf{Task 1:} The \texttt{gridRender.sh} script just creates a proxy and copies the files to the appropriate instancens. One interesting thing to mention is the grid resource \texttt{login.leo1.uibk.ac.at}, because for running jobs on this resource the addition \texttt{/jobmanager-sge} is compulsory.
\item \textbf{Task 2:} The \texttt{gridRender.sh} script executes a \texttt{globus}-job for each resource with a specific subset of frames. The merging of the images is done on the \texttt{karwendel} machine, after all files have been copied to it.
\item \textbf{Task 3:} The \texttt{gridRender.sh} script takes timestamps before and after each grid-resource execution. The results are described in the next section.

\end{itemize}


%----------------------------------------------------------------------------------------
%	SECTION 3
%----------------------------------------------------------------------------------------

\section{Results}

The environment contains 3 grid resources, 2 were used for measurements:

\begin{itemize}
\item \texttt{kar\-wen\-del.\-dps.\-uibk.\-ac.\-at} 
\item \texttt{login.\-leo1.\-uibk.\-ac.\-at/\-jobmanager-sge}
\end{itemize}

Only one processor of each machine got a grid job, so that there is a better comparison between the resources. For faster results it would be much better to make use of all processors.
\\
\\
Measurement were made for 8, 16, 32, 64 and 128 frames. Table~\ref{tab:measurements} shows the measurement results, where T$_k$ is the execution time of the \texttt{karwendel} machine and T$_L$ the execution time of the \texttt{leo1} machine (both without copying the results to the image merge machine).

\begin{table}[htbp]
\centering
\begin{tabular}{ | c | c | c | p{2.5cm} | p{2.5cm} |}
\hline
\textbf{Frames} & \textbf{T$_K$} in s & \textbf{T$_{L}$} in s & \textbf{T$_{K}$/(Frames)}/2 in s & \textbf{T$_{L}$/(Frames)}/2 in s \\
\hline \hline
8 & 81,47 & 91,14 & 20,37 & 22,79 \\
\hline
16 & 141,70 & 171,25 & 17,71 & 21,41 \\
\hline
32 & 272,36 & 301,54 & 17,02 & 18,85 \\
\hline
64 & 543,87 & 602,28 & 17,00 & 18,82 \\
\hline
128  & 1066,39 & 1153,45 & 16,67 & 18,02 \\
\hline
%32 & 32 & 582,99 & 75,93 & 7,68 & 0,19 \\
%\hline
%64 & 40 & 1161,30 & 141,94 & 8,18 & 0,20 \\
%\hline
%128 & 40 & 2319,09 & 277,01 & 8,37 & 0,21 \\
%\hline
\end{tabular}
\caption{Measurements}
\label{tab:measurements}
\end{table}

The \texttt{karwendel} machine processor seems to be faster than the \texttt{leo1} machine one. While the number of frames increases the time/frame decreases for both machines. This is probably due to the relatively lesser job submitting overhead.
\\
\\
For a load imbalance as little as possible the time/frame and the measurement with the largest number of frames (smallest uncertainty) are crucial. The best distribution for 128 frames for the same environment and the same condition would be:
\\
\\
\(karwendel = \frac{18,02}{16,67 + 18,02} = 52\% \)
\\
\\
\(leo1      = \frac{16,67}{16,67 + 18,02} = 48\% \)
\\
\\
Of course the utilization of all processors of all machines would reveal another result.
%Because of this reaction, the required ratio is the atomic weight of magnesium: \SI{16.00}{\gram} of oxygen as experimental mass of Mg: experimental mass of oxygen or $\frac{x}{1.31}=\frac{16}{0.87}$ from which, $M_{\ce{Mg}} = 16.00 \times \frac{1.31}{0.87} = 24.1 = \SI{24}{\gram\per\mole}$ (to two significant figures).

%----------------------------------------------------------------------------------------
%	SECTION 4
%----------------------------------------------------------------------------------------

%\section{Results and Conclusions}

%The atomic weight of magnesium is concluded to be \SI{24}{\gram\per\mol}, as determined by the stoichiometry of its chemical combination with oxygen. This result is in agreement with the accepted value.

%\begin{figure}[h]
%\begin{center}
%\includegraphics[width=0.65\textwidth]{placeholder} % Include the image placeholder.png
%\caption{Figure caption.}
%\end{center}
%\end{figure}
%
%----------------------------------------------------------------------------------------
%	SECTION 5
%----------------------------------------------------------------------------------------

%\section{Discussion of Experimental Uncertainty}

%The accepted value (periodic table) is \SI{24.3}{\gram\per\mole} \cite{Smith:2012qr}. The percentage discrepancy between the accepted value and the result obtained here is 1.3\%. Because only a single measurement was made, it is not possible to calculate an estimated standard deviation.

%The most obvious source of experimental uncertainty is the limited precision of the balance. Other potential sources of experimental uncertainty are: the reaction might not be complete; if not enough time was allowed for total oxidation, less than complete oxidation of the magnesium might have, in part, reacted with nitrogen in the air (incorrect reaction); the magnesium oxide might have absorbed water from the air, and thus weigh ``too much." Because the result obtained is close to the accepted value it is possible that some of these experimental uncertainties have fortuitously cancelled one another.


%----------------------------------------------------------------------------------------
%	BIBLIOGRAPHY
%----------------------------------------------------------------------------------------

%\bibliographystyle{apalike}

%\bibliography{sample}

%----------------------------------------------------------------------------------------


\end{document}
